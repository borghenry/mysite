\documentclass[11pt,]{article}
\usepackage[sc, osf]{mathpazo}
\usepackage{amssymb,amsmath}
\usepackage{ifxetex,ifluatex}
\usepackage{fixltx2e} % provides \textsubscript
\ifnum 0\ifxetex 1\fi\ifluatex 1\fi=0 % if pdftex
  \usepackage[T1]{fontenc}
  \usepackage[utf8]{inputenc}
\else % if luatex or xelatex
  \ifxetex
    \usepackage{mathspec}
  \else
    \usepackage{fontspec}
  \fi
  \defaultfontfeatures{Ligatures=TeX,Scale=MatchLowercase}
\fi
% use upquote if available, for straight quotes in verbatim environments
\IfFileExists{upquote.sty}{\usepackage{upquote}}{}
% use microtype if available
\IfFileExists{microtype.sty}{%
\usepackage{microtype}
\UseMicrotypeSet[protrusion]{basicmath} % disable protrusion for tt fonts
}{}
\usepackage[margin=1in]{geometry}




\setlength{\emergencystretch}{3em}  % prevent overfull lines
\providecommand{\tightlist}{%
  \setlength{\itemsep}{0pt}\setlength{\parskip}{0pt}}
\setcounter{secnumdepth}{0}
% Redefines (sub)paragraphs to behave more like sections
\ifx\paragraph\undefined\else
\let\oldparagraph\paragraph
\renewcommand{\paragraph}[1]{\oldparagraph{#1}\mbox{}}
\fi
\ifx\subparagraph\undefined\else
\let\oldsubparagraph\subparagraph
\renewcommand{\subparagraph}[1]{\oldsubparagraph{#1}\mbox{}}
\fi

% Now begins the stuff that I added.
% ----------------------------------

% Custom section fonts
\usepackage{sectsty}
\sectionfont{\rmfamily\mdseries\large\bf}
\subsectionfont{\rmfamily\mdseries\normalsize\itshape}


% Make lists without bullets
\renewenvironment{itemize}{
  \begin{list}{}{
    \setlength{\leftmargin}{1.5em}
  }
}{
  \end{list}
}


% Make parskips rather than indent with lists.
\usepackage{parskip}
\usepackage{titlesec}
\titlespacing\section{0pt}{12pt plus 4pt minus 2pt}{4pt plus 2pt minus 2pt}
\titlespacing\subsection{0pt}{12pt plus 4pt minus 2pt}{4pt plus 2pt minus 2pt}

% Use fontawesome. Note: you'll need TeXLive 2015. Update.


% Fancyhdr, as I tend to do with these personal documents.
\usepackage{fancyhdr,lastpage}
\pagestyle{fancy}
\renewcommand{\headrulewidth}{0.0pt}
\renewcommand{\footrulewidth}{0.0pt}
\lhead{}
\chead{}
\rhead{}
\lfoot{
\cfoot{\scriptsize  Enrico Borghetto - CV }}
\rfoot{\scriptsize \thepage/{\hypersetup{linkcolor=black}\pageref{LastPage}}}

% Always load hyperref last.
\usepackage{hyperref}
\PassOptionsToPackage{usenames,dvipsnames}{color} % color is loaded by hyperref

\hypersetup{unicode=true,
            pdftitle={Enrico Borghetto:  CV (Curriculum Vitae)},
            pdfauthor={Enrico Borghetto},
            colorlinks=true,
            linkcolor=blue,
            citecolor=Blue,
            urlcolor=blue,
            breaklinks=true, bookmarks=true}
\urlstyle{same}  % don't use monospace font for urls

\begin{document}


\centerline{\huge \bf Enrico Borghetto}

\vspace{2 mm}

\hrule

\vspace{2 mm}

\moveleft.5\hoffset\centerline{FCT Investigator}

\moveleft.5\hoffset\centerline{ \emph{E-mail:} \href{mailto:}{\tt \href{mailto:enrico.borghetto@fcsh.unl.pt}{\nolinkurl{enrico.borghetto@fcsh.unl.pt}}} \hspace{1 mm}     \emph{Web:} \href{http://enricoborghetto.netlify.com}{\tt enricoborghetto.netlify.com}   }

\vspace{2 mm}

\hrule


Last updated: 2018-06-20

\hypertarget{work-address}{%
\subsection{WORK ADDRESS}\label{work-address}}

Interdisciplinary Centre of Social Sciences - CICS.NOVA\\
NOVA University of Lisbon\\
Avenida de Berna, n. 26-C, 1069-061 Lisboa

\hypertarget{academic-appointments}{%
\subsection{ACADEMIC APPOINTMENTS}\label{academic-appointments}}

\begin{itemize}
\item
  \textbf{2015-PRESENT} FCT investigator at CICS.NOVA, FCSH\\
  University NOVA of Lisbon, Portugal
\item
  \textbf{2013-2015} Postdoctoral Researcher at CESNOVA, FCSH, with a
  project entitled ``The policy agenda at the time of crisis in Europe's
  periphery'' (Grant funded by the Fundação para a Ciência e a
  Tecnologia, SFRH/BPD/89968/2012)\\
  University NOVA of Lisbon, Portugal
\item
  \textbf{2008-2012} Postdoctoral Researcher, Department of Social and
  Political Studies with a project entitled ``The europeanisation of
  executive-legislative relations: national parliaments and Community
  legislation''\\
  University of Milan, Milano, Italy
\end{itemize}

\hypertarget{education}{%
\subsection{EDUCATION}\label{education}}

\begin{itemize}
\item
  \textbf{2003-2007} Ph.D.~in Political Studies\\
  Graduate School in Social, Economic and Political Sciences of the
  University of Milan\\
  Dissertation: `Non-compliance with the transposition deadlines of EU
  directives: the Italian case. Explaining transposition of EU
  directives into Italian legislation' (written in English and discussed
  on 17/07/2007) Supervisor: Prof.~Marco Giuliani, Examination Committee
  Chair: Prof.~Maurizio Ferrera
\item
  \textbf{1998-2003} University Degree in International and Diplomatic
  Sciences (110/110 magna cum laude)\\
  University of Bologna (Forlì), Faculty of International and
  Diplomatic Sciences\\
  Dissertation: ``The accountability of international organisations: the
  Principal -Agent model applied to the WTO and the IMF''. Supervisor:
  Prof.~Salvatore Vassallo
\item
  \textbf{1993-1998} High School Diploma in Foreign Languages (English,
  French and German)\\
  Liceo Linguistico sperimentale A.Canova, Treviso
\end{itemize}

\hypertarget{research-experience}{%
\subsection{RESEARCH EXPERIENCE}\label{research-experience}}

\begin{itemize}
\item
  \textbf{2015-PRESENT} Principal investigator ``Portuguese Parliament:
  Agenda-setting and Law-making'' Project financed by the Fundação
  para a Ciência e a Tecnologia (IF/00382/2014)
\item
  \textbf{2015-PRESENT} Researcher in the project ``Democracy in times
  of crisis: Power and Discourse in a three-level game'', Project
  financed by the Fundação para a Ciência e a Tecnologia
  (PTDC/IVC-CPO/2247/2014) URL: \url{https://goo.gl/ExHizi}
\item
  \textbf{2015-PRESENT} Researcher in the project ``Crisis, Political
  Representation and Democratic Renewal: The Portuguese case in the
  Southern European context'', Project financed by the Fundação para a
  Ciência e a Tecnologia (PTDC/IVC-CPO/3098/2014) URL:
  \url{https://goo.gl/HMBMaK}
\item
  \textbf{2014-PRESENT} Principal investigator in the Portuguese team
  participating in the European project ``Political Leaders in European
  Cities'' (Coordinated by the University of Florence) This project
  deals with the role of Mayors and the transformation of political
  representation and careers at the local level.
\item
  \textbf{2013-2015} Associate researcher for the project ``Public
  preferences and Policy decision-making: a comparative analysis''
  Project financed by the Fundação para a Ciência e a Tecnologia
  (PTDC/IVC-CPO/3921/2012)\\
  This project aims at increasing our understanding of democratic party
  government and the relationship between the public opinion and the
  agenda-setting process in Portugal.
\item
  \textbf{2010-PRESENT} Co-manager of the ILMA (Italian-Law Making
  Database)\\
  ILMA originated from an idea of a group of scholars affiliated with
  the Department of Social and Political Studies at the Universitá degli
  Studi di Milano, and it is one of the scientific products of the
  Center for the Observation of Legislatures (COoL). It is a relational
  database that combines information on Italian legislation, roll calls
  and political elites. It currently covers six Italian legislatures
  (1987-2008).
\item
  \textbf{2010-2012} Member of the research unit of the University of
  Milan in the project -Institutional agenda setting: Actors, time,
  information.- co-financed by the Italian Ministry for Research and
  Higher Education (PRIN 2009, Protocol 2009TPW4NL\_002, Principal
  Investigator: prof. Marco Giuliani).
\item
  \textbf{2009-PRESENT} Researcher and one of the Principal
  Investigators for the project ``Italian Policy Agendas''\\
  This is a joint research project involving the Universities of Milan,
  Siena and Malta (\url{http://italianpolicyagendas.weebly.com}). It
  aims at exploring the evolution of issue attention in various
  institutions (media, political parties, parliament, government) and
  its impact on public policy outcomes using an agenda-setting
  perspective. It is part of a larger international research network:
  the ``Comparative Agendas Project''.
\item
  \textbf{2008-2012} Researcher in the Italian team participating in the
  European project ``Delors' Myth: The scope and impact of
  Europeanization on law production''. The project aims are: 1)
  developing new methods to quantitatively measure the Europeanization
  of national public policies (i.e.~the scope and extent national
  policies are shaped by EU law and policy); 2) analysing the impact of
  EU policy-making on the relationship between government and parliament
  in each of the nine countries under study.
\item
  \textbf{Feb/Mar 2012} Visiting Fellow Mannheim Centre for European
  Social Research (MZES)\\
  University of Mannheim, Mannheim, Germany
\item
  \textbf{Jun/Aug 2010} Visiting Fellow, Center for American Politics
  and Public Policy (CAPPP)\\
  University of Washington, Seattle, USA
\item
  \textbf{2006} Research assistant at the Research Unit on European
  Governance (URGE)\\
  Collegio Carlo Alberto di Moncalieri, Torino, Italy\\
  Key areas of responsibility: review of the literature relevant for
  URGE projects; assistance in the preparation of manuscripts and
  reports; management of events organized by URGE.
\item
  \textbf{Jan/Aug 2005} Research assistant for the project: ``Compliance
  with EU Law: Explaining the Transposition of EU Directives''
  (Coordinator: Prof.~Fabio Franchino)\\
  University College London, Department of Political Science, London,
  UK\\
  Key areas of responsibility: data collection and maintenance of the
  database on national transposition of EU directives; review of the
  literature on EU Compliance; contribution to joint publications.
\item
  \textbf{Oct/Dec 2004} Research assistant for the project ``Political
  representation and legislative process in the regions and the
  autonomous provinces in Italy'' (Coordinator Prof.~Salvatore
  Vassallo)\\
  Istituto Carlo Cattaneo and Universitá di Bologna, Bologna, Italy\\
  Key areas of responsibility: assistance in drafting the questionnaire;
  interviews with over 30 members of the Lombardia Regional Council.
\item
  \textbf{Nov 2002-Mar 2003} One World Trust (OWT), London\\
  Internship at OWT, think tank based at the House of Commons in London.
  OWT undertakes research and education projects to promote global
  democracy. This internship was an integral part of my undergraduate
  thesis preparation.
\end{itemize}

\hypertarget{teaching-experience}{%
\subsection{TEACHING EXPERIENCE}\label{teaching-experience}}

\begin{itemize}
\item
  \textbf{21/02/2018-30/05/2018} Politics in challenging times.
  Populism, protests and the crisis of representation after the Great
  Recession, Master Course, Nova University of Lisbon, 10 ECTS
\item
  \textbf{05/02/2018-08/02/2018} The Essentials of Quantitative
  Research. Data Analysis in R\\
  Lisbon Winter School in Research Skills and Methods, NOVA University
  of Lisbon, 2 ECTS
\item
  \textbf{30/01/2017-02/02/2017} The Essentials of Quantitative
  Research. Data Analysis in R\\
  Lisbon Winter School in Research Skills and Methods, NOVA University
  of Lisbon, 2 ECTS
\item
  \textbf{29/8/2016-02/09/2016} The Essentials of Data Analysis in R\\
  Lisbon Summer School, NOVA University of Lisbon
\item
  \textbf{02/03/2016-25/05/2016} The political and social consequences
  of the Great Recession in Southern Europe, Master Course, Nova
  University of Lisbon, 10 ECTS
\item
  \textbf{07/03/2016-10/03/2016} The Essentials of Quantitative
  Research. Data Analysis in R\\
  Lisbon Winter School in Research Skills and Methods, NOVA University
  of Lisbon, 2 ECTS
\item
  \textbf{09/03/2015-12/03/2015} The Essentials of Quantitative
  Research. Data Analysis in R\\
  Lisbon Winter School in Research Skills and Methods, NOVA University
  of Lisbon, 2 ECTS
\item
  \textbf{10/02/2014-13/02/2014} The Essentials of Quantitative
  Research. Data Analysis in R\\
  Lisbon Winter School in Research Skills and Methods, NOVA University
  of Lisbon, 2 ECTS
\item
  \textbf{03/05/2013-08/05/2013} Introduction to Event History Analysis
  with STATA\\
  Graduate School of the University of Milan, Milan, May 2013
\item
  \textbf{13/02/2013-16/02/2013} Introduction to Event History
  Analysis\\
  Lisbon Winter School in Research Skills and Methods, NOVA University
  of Lisbon, 2 ECTS
\item
  \textbf{2010-2012} Teaching assistant for the postgraduate course
  ``Democratic Governance and Public Administration: The European
  Union'' (Prof.Fabio Franchino)\\
  University of Milan, Faculty of Political Science, Master in Economics
  and Political Science\\
  Key areas of responsibility: collaboration to course design,
  supervising students, guest lecturer
\item
  \textbf{2006-2011} Teaching assistant for the undergraduate course
  ``Introduction to Political Science'' (Prof.Fabio Franchino)\\
  University of Milan, Faculty of Political Science, Degree in
  International studies and European institutions\\
  Key areas of responsibility: preparing and grading exams, supervising
  students, guest lecturer.
\end{itemize}

\hypertarget{grants-awarded}{%
\subsection{GRANTS AWARDED}\label{grants-awarded}}

\begin{itemize}
\item
  Part of the research team located in Milan university awarded the PRIN
  funding in both 2007 (prot. scrwt4) and 2009 (prot. 2009TPW4NL\_002).
  PRIN= Research Programs of National Relevance awarded by the Italian
  Ministry of Education and Research
\item
  Research Unit on European Governance (URGE) - 1-year research grant
\item
  PhD scholarship (2003-2007) by the Italian Ministry of Education and
  Research
\item
  6-month scholarship in 2002 to conduct undergraduate thesis research
  in the UK (awarded by the University of Bologna-Forlì)
\item
  9-month Erasmus scholarship in 2000, University of Sussex, UK
\end{itemize}

\hypertarget{publications}{%
\subsection{PUBLICATIONS}\label{publications}}

\begin{itemize}
\item
  with Marco Lisi (Forthcoming, 2018) ``Productivity and re-selection in
  a party-based environment: evidence from the Portuguese case''
  Parliamentary Affairs
\item
  (Forthcoming, 2018) ``Delegated decree authority in a parliamentary
  system. The exercise of legislative delegation in Italy (1987-2013)''
  Journal of legislative Studies
\item
  (Forthcoming, 2018) ``Anti-populism'', in F.José Eduardo (Ed)
  Dicionário dos Antis. A Cultura Portuguesa em Negativo, Imprensa
  Nacional-Casa da Moeda
\item
  with Federico Russo (2018) From agenda setters to agenda takers? The
  determinants of party issue attention in times of crisis. Party
  Politics 24, 65-77
\item
  with Ivan Kopric, Eva Marín Hlynsdóttir, Jasmina Dzinic (2018),
  Institutional Environments and Mayors' Role Perceptions, In H. Heinelt
  et al. (eds.), Political Leaders and Changing Local Democracy,
  Palgrave Macmillan pp.149-173
\item
  with Francesco Visconti and Marco Michieli (2017) ``Government Agenda
  Setting in Italian Coalitions, Testing the partisan hypothesis using
  Italian investiture speeches 1979-2014'' Italian Journal of Public
  Policies (2): 193-220
\item
  , (2016), Book review ``Crise Económica, Políticas de Austeridade e
  Representação Política, Lisboa, Assembleia da República - Divisão de
  edições, 2015''. Análise Social, 219, li (2.º), pp.~462-466.
\item
  , (2015), ``Questioning the government in time of crisis. An analysis
  of Question Time in Spain'', In J. Preunkert and G. Vobruba (Eds)
  Aftermath. Beyond the Crisis of the European Currency, Lisbon:
  Colibri, 91-116
\item
  , (2015), ``Challenging Italian Centralism through the Vertical Shift
  of Competences to the Subnational and Supranational Levels.''
  Contemporary Italian Politics 7 (1): 58-75.
\item
  with Marcello Carammia (2015) ``The Influence of Coalition Parties on
  Executive Agendas in Italy (1983-2008).'' In The Challenge of
  Coalition Government: The Italian Case, eds. Nicolò Conti and
  Francesco Marangoni. Abingdon: Routledge, 36-57.
\item
  with Francesco Visconti (2015) ``Governing by Revising. A Study on
  Post-Enactment Policy Change in Italy.'' In The Challenge of Coalition
  Government: The Italian Case, eds. Nicolò Conti and Francesco
  Marangoni. Abingdon: Routledge, 106-27.
\item
  , (2014) Legislative processes as sequences: exploring the temporal
  dimension of law-making by means of sequence analysis, International
  Review of Administrative Sciences. 80:3, pp.553-76
\item
  with Lars Mader (2014) EU Law Revisions and Legislative Drift,
  European Union Politics.15:2, pp.171-191
\item
  with Marcello Carammia and Francesco Zucchini (2014). The impact of
  party priorities on Italian law-making from The First to the Second
  Republic, In C.Green Pedersen and S.Walgrave (Eds.) Agenda Setting,
  Policies, and Political Systems, Chicago: Chicago University Press,
  pp.164-182
\item
  , (2013) Keeping the pace with Europe: Non-compliance with the
  transposition deadlines of EU directives in the Italian case, Novi
  Ligure: Epoke edizioni, ISBN: 978-88-98014-13-2
\item
  with Luigi Curini, Marco Giuliani, Alessandro Pellegata and Francesco
  Zucchini (2012) Italian Law-Making Archive (ILMA): A new tool for the
  analysis of the Italian legislative process, Rivista Italiana di
  Scienza Politica, 3 pp.~481-502
\item
  with Marco Giuliani (2012) A Long Way to Tipperary: Time in the
  Italian Legislative Process 1987-2008, South European Society and
  Politics , 17:1 pp.23-44
\item
  with Marco Giuliani and Francesco Zucchini (2011) ``Leading
  governments and unwilling legislators. The European Union and the
  Italian Law making (1987-2006)'', in S. Brouard, O. Costa and T.König
  (Eds), The Europeanization of Domestic Legislatures, New
  York:Springer, pp.109-130
\item
  with Fabio Franchino (2010) ``The Role of Subnational Authorities in
  the Implementation of EU Directives'', Journal of European Public
  Policy, 17:6, pp.~759 - 780
\item
  with Marcello Carammia (2010) ``L'analisi comparata delle agenda
  politiche: il Comparative Agendas Project'', Rivista Italiana di
  Scienza Politica, n.2, pp.301-315
\item
  with Marco Giuliani and Francesco Zucchini (2009) ``Quanta Bruxelles
  c'è a Roma? L'europeizzazione della produzione normativa italiana.'',
  Rivista italiana di Politiche Pubbliche, n.1, pp.135-162
\item
  with Fabio Franchino and Daniela Giannetti (2006) ``Complying with the
  Transposition Deadlines of EU Directives: Evidence from Italy'',
  Rivista Italiana di Politiche Pubbliche, n.5, pp.~7-38.
\end{itemize}

\hypertarget{other-research-outputs}{%
\subsection{OTHER RESEARCH OUTPUTS}\label{other-research-outputs}}

\begin{itemize}
\item
  with Elisabetta De Giorgi and Marco Lisi (2014) Government failure,
  opposition success? Parties electoral performance in Portugal and
  Italy at the time of the crisis.Jean Monnet Occasional Paper,
  No.05/2014, Institute for European Studies (Malta).
\item
  The respect of transposition deadlines in Italy: do political
  priorities matter? URGE Working paper 7/2007
\end{itemize}

\hypertarget{work-in-progress}{%
\subsection{WORK IN PROGRESS}\label{work-in-progress}}

\begin{itemize}
\item
  with Marcello Carammia and Shaun Bevan. Changing the Transmission
  Belt: The Programme-to-Policy Link in Italy between the First and
  Second Republic. {[}submitted{]}
\item
  with Marco Lisi. Populism, blame shifting and the crisis:
  communication strategies in Portuguese political parties
  {[}submitted{]}
\item
  with Derek Epp. Economic inequality and legislative agendas in Europe
  {[}submitted{]}.
\item
  with Alessandro Pellegata. Exploring bill winnowing in the Italian
  Chamber of Deputies. Working paper
\item
  with André Freire and José Santana Lopes. Constituency
  Characteristics, Expertise, Civil Society Links and Parliamentary
  Questions: Evidence from Portugal. Working paper
\item
  with Marco Lisi. Parliamentary questions and individual representation
  in a party-dominated environment: evidence from Portugal. Working
  paper
\item
  with Ana Belchior. The transmission of policy agendas. The effect of
  party competition on the gap between electoral priorities and policy
  outputs. Working paper
\item
  with Elisabetta de Giorgi. The Five-Star Movement in Parliament: a
  Truly New Kind of Parliamentary Opposition?
\item
  with Ana Belchior. The determinants of the executive weekly agenda in
  Portugal: party mandates vs media attention
\end{itemize}

\hypertarget{conference-papers-a-selection}{%
\subsection{CONFERENCE PAPERS (A
SELECTION)}\label{conference-papers-a-selection}}

\begin{itemize}
\item
  ``Economic inequality and legislative agendas in Europe'' 18-20 April
  2018, Conference of the Portuguese Political Science Association,
  Braga
\item
  ``Constituency characteristics, civil society links and topic
  selection by individual legislators: Evidence from Portugal'' 10-14
  April 2018, ECPR Joint Session, Nicosia
\item
  ``The determinants of the executive weekly agenda in Portugal: party
  mandates vs media attention'' 6th December, 2017, University of Texas,
  Austin
\item
  ``The Five-Star Movement in Parliament: a Truly New Kind of
  Parliamentary Opposition?'', 14-16 September 2017, Annual Conference
  of the Italian Political Science Association, Urbino
\item
  ``Constituency Characteristics, Expertise, Civil Society Links and
  Parliamentary Questions: Evidence from Portugal.'' 6-9 September 2017,
  ECPR General Conference, Oslo
\item
  ``Populism, blame shifting and the crisis: communication strategies in
  Portuguese political parties'', 6-9 September 2017, ECPR General
  Conference, Oslo
\item
  ``The determinants of the executive weekly agenda in Portugal: party
  mandates vs media attention'' June 15-17, 2017, Annual Conference of
  the Comparative Agendas network, Edinburgh
\item
  ``Delegated decree authority in a parliamentary system. The exercise
  of legislative delegation in Italy'' (1987-2013), 15-17 September
  2016, SISP Annual Conference, University of Milan
\item
  ``Parliamentary Questions and Individual Representation in a
  Party-Dominated Environment: Evidence from Portugal'', July 23-28,
  2016, IPSA World Congress - International Political Science
  Association, Poznan, Poland
\item
  ``Parliamentary Questions and Individual Representation in a
  Party-Dominated Environment: Evidence from Portugal'', June 30 to July
  2 2016, ECPR Standing group on parliaments, University of Munich
\item
  ``Between contraction of agendas and issue expansion: The impact of
  the Euro crisis on partiesan issue attention'', 27-29 June 2016,
  Annual Conference Comparative Agendas Project, University of Geneva.
\item
  ``The determinants of party issue attention in time of crisis: from
  agenda setters to agenda takers?'', 10-12 March, 2016, Portuguese
  Association of Political Science, Nova University of Lisbon.
\item
  ``Challenging the government in parliament: an analysis of question
  time in time of austerity'', Workshop - O Lusitanismo Italiano" 17-18
  December, 2015, University of Bologna.
\item
  ``Challenging the government in parliament: an analysis of question
  time in time of austerity'', Workshop - Party Competition and
  Political Representation in Crisis: A Comparative Perspective" 24-25
  September, 2015, at the European University Institute, Florence.
\item
  ``The transmission of policy agendas. The effect of party competition
  on the gap between electoral priorities and policy outputs'', 22-24
  June 2015, Annual Conference Comparative Agendas Project, University
  of Lisbon
\item
  ``The impact of party priorities on Portuguese legislative
  activities'', 22-24 June 2015, Annual Conference Comparative Agendas
  Project, University of Lisbon
\item
  ``Questioning the government in time of crisis. A comparative analysis
  of the content of parliamentary questions in Italy, Portugal and
  Spain'', 20-21 November 2014, Workshop ``Policy-Making in Hard Times:
  Southern European Countries in a Comparative Perspective'', Institut
  Barcelona d'Estudis Internacionals (IBEI) \& Instituto Carlos III-Juan
  March de Ciencias Sociales (IC3JM), Madrid
\item
  ``Challenging Italian centralism: the vertical shift of competences to
  the subnational and supranational level''. 11-13 September 2014, SISP
  Annual Conference, University of Perugia
\item
  ``Exploring Bill Winnowing in the Italian Chamber of Deputies
  (1996-2012)'', 11-12 July 2014, Workshop ``Parliamentary Scrutiny of
  EU Politics'' Heidelberg University
\item
  ``Agenda-setting in times of crisis'', 24-26 July 2014, Workshop
  ``Changes to political representation in Southern Europe in times of
  crisis'', University of Nottingham, UK
\item
  ``Questioning the government in times of crisis'', 14-16 April 2014,
  Biannual Conference Portuguese Political Science Association, Coimbra
\item
  ``The political fallouts of the Great Depression'', 28 April 2014,
  ASEU - Jean Monnet Module on Agenda-Setting in the European Union,
  University of Malta, Malta
\item
  ``Tracking attention to issues as a way to learn about political
  systems'', 23 October 2013, INESD, Instituto de Engenharia de Sistemas
  e Computadores, Investigação e Desenvolvimento, Lisboa
\item
  ``Government Agenda-Setting in Italian Coalitions: An analysis of
  investiture speeches in Italy 1979-2013'', 12-14 September 2013, SISP
  Annual Conference, Florence
\item
  ``The Influence of Coalition Parties on Governments' Legislative
  Agendas in Italy Between the First and Second Republic''. 27-29 June
  2013, Annual Conference Comparative Agendas Project, Antwerp
\item
  ``The evolution of Italian law. A study on post-enactment policy
  change between the 1st and 2nd Italian Republic'', 13-15 September
  2012, SISP Annual Conference, Rome
\item
  ``Italian: Law-Making Archive: A New Tool For the Analysis Of the
  Italian Legislative Process'', 6 July 2012, The Law Factory, (Sciences
  Po, CEE, Regards Citoyens, Médialab), Paris
\item
  ``Legislative processes as sequences: exploring the temporal dimension
  of law-making by means of sequence analysis'', 10-15 April 2012, ECPR
  Joint Session 2012, Antwerp
\item
  ``Legislative processes as sequences: exploring the temporal dimension
  of law-making by means of sequence analysis'', 6-8 June 2012, Lausanne
  Conference on Sequence Analysis, Lausanne
\item
  ``The evolution of EU law: analysing the longevity of EU
  legislation'', 16-18 June 2011, 2011 EPSA General Conference, Dublin
\item
  ``The impact of party priorities on Italian law-making from the First
  to the Second Republic'', 17-18 June 2010, 2010 Comparative Policy
  Agendas Conference, University of Washington, Seattle
\item
  ``The impact of party priorities on Italian law-making from the First
  to the Second Republic'', 20-21 May 2010, ``Political Parties and
  Comparative Policy Agendas: an ESF Workshop on Political Parties and
  their Positions, and Policy Agendas'', University of Manchester, UK
\item
  ``Exploring why Italian executives fail to exercise the legislative
  powers they are delegated'', 14-15 January 2010, the ``International
  conference on democracy as idea and practice'', University of Oslo,
  Norway
\item
  ``Exploring why Italian executives do not exercise the legislative
  powers they are delegated'', 9-12 September 2009, 2009 ECPR General
  Conference, Potsdam
\item
  ``Leading governments and unwilling legislators. The European Union
  and the Italian Law making (1987-2006)'', Hague, 17 June 2009,
  ``Delors' Myth Workshop 2: The scope and impact of Europeanization of
  law production''
\item
  ``The regional dimension of EU policy implementation'', 25-27
  September 2008, ``Fourth Pan-European Conference on EU Politics
  (ECPR)'', Riga
\item
  ``The pace of the legislative process. A diachronic analysis of the
  Italian legislature (1996-2006)'', 4-6 September 2008, 2008 SISP
  (Italian Political Science Association) annual conference``, Pavia
\item
  ``The timely transposition of EU directives in Italy: do political
  priorities matter?'', 11-16 April 2008, ECPR Joint sessions of
  workshops, Workshop ``The Long Arm of EU Law'', Rennes
\end{itemize}

\hypertarget{additional-training}{%
\subsection{ADDITIONAL TRAINING}\label{additional-training}}

\begin{itemize}
\item
  \textbf{Mar 2017} ECPR Winter School in Methods and Techniques,
  Bamberg, Germany\\
  One-week course in `Time-series Analysis'.
\item
  \textbf{Feb 2012} ECPR Winter School in Methods and Techniques,
  Vienna, Austria\\
  One-week course in `Sequence Analysis'.
\item
  \textbf{Jun 2010} Five day workshop on sequence analysis for political
  scientists Tools for text workshop, University of Washington, Seattle,
  US\\
  Two-day workshop covering a range of content analysis approaches:
  Manual Annotation; Unsupervised Learning; Supervised Learning;
  Dimensional Scaling.
\item
  \textbf{Mar 2010} Oxford Spring School, Oxford, UK\\
  One-week course in ``Modern Regression''.
\item
  \textbf{Jun 2009} Empirical Implications of Theoretical Models,
  Mannheim, Germany\\
  The two-week summer school focused on modelling techniques and methods
  of empirical testing theoretical models in the social science.
\item
  \textbf{Jul 2005} Essex Summer School in Social Science Data Analysis,
  UK\\
  The two-week summer school focused on Maths for Social Scientists,
  Survival Analysis, Mixing Methods
\end{itemize}

\hypertarget{other-professional-activities}{%
\subsection{OTHER PROFESSIONAL
ACTIVITIES}\label{other-professional-activities}}

\hypertarget{academic-service}{%
\subsubsection{Academic service}\label{academic-service}}

\begin{itemize}
\item
  Co-chair for the 8th annual International Conference of the
  Comparative Agendas Project, Nova University of Lisbon/ISCTE, 22-24
  June 2015.
  \url{http://www.comparativeagendas.net/pages/2015-conference-in-lisbon-portugal}
\item
  Organizer for the Pedro Hispano Winter School in Research Skills and
  Methods 2014 and 2015, Lisboa
\item
  Chair and discussant in the Conference ``Transnational public
  participation and social movement activism'', Conference FCSH/CIES,
  November 2013
\item
  Chair and discussant in the panel ``The comparative analysis of policy
  agendas'' organized for the 2011 SISP (Italian Political Science
  Association) annual conference, Palermo University, Palermo, 8-10
  September 2011
\item
  Member of the organizing committee for the 4th annual International
  Conference of the Comparative Agendas Project, University of Catania,
  23-25 June 2011
\item
  Chair and discussant in the panel ``Agenda-setting e policy making''
  organized for the 2009 SISP (Italian Political Science Association)
  annual conference, LUISS University, Roma, 17-19 September 2009
\item
  Discussant in the ``Project Colloquium Seminars'', Graduate School of
  the University of Milan (since 2009)
\end{itemize}

\hypertarget{translation}{%
\subsubsection{Translation}\label{translation}}

\begin{itemize}
\tightlist
\item
  Il Mulino Publishing House, Bologna\\
  Translation from English to Italian of the book ``The government and
  politics of the European Union'' by N.Nugent, London: MacMillan (2006,
  6° ed.)
\end{itemize}

\hypertarget{peer-review}{%
\subsubsection{Peer Review}\label{peer-review}}

\begin{itemize}
\tightlist
\item
  European Union Politics\\
\item
  Journal of European Political Research\\
\item
  Political Studies\\
\item
  Comparative political studies\\
\item
  Rivista Italiana di Scienza Politica\\
\item
  Publius\\
\item
  Journal of Legislative Studies\\
\item
  Analise Social
\end{itemize}

\hypertarget{language-skills}{%
\subsection{LANGUAGE SKILLS}\label{language-skills}}

\begin{itemize}
\tightlist
\item
  Proficient user in both written and spoken English (ESOL Certificate
  in Advanced English) and Portuguese.
\item
  School knowledge of Spanish, French and German.
\end{itemize}

\hypertarget{special-skills}{%
\subsection{SPECIAL SKILLS}\label{special-skills}}

\begin{itemize}
\tightlist
\item
  R, Rmarkdown
\item
  Ms Office
\item
  SQL
\item
  SPARQL
\item
  Python
\end{itemize}

\end{document}
